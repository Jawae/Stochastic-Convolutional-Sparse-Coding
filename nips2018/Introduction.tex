\section{Introduction}
Convolutional Sparse Coding (CSC) has been demonstrated to be an effective tool to learn translational invariant dictionary from a number of training signals, and shows its applications in recent researches about neural and brain information processing~\cite{jas2017learning,peter2017sparse}, as well as in a variety of image processing tasks, for instance, image inpainting~\cite{heide2015fast}, super-resolution~\cite{gu2015convolutional}, high dynamic range imaging~\cite{serrano2016convolutional}, and high-dimensional signal reconstructions~\cite{choudhury2017consensus,bibi2017high}. CSC differs from conventional sparse coding by formulating the signals as a summation of a set of convolutions on dictionary filters and sparse codes, rather than linearly adding up those filters. Alternatively, sparse coding, as a patch-based approach, learns filters from partitioned local structures, and this partition manipulation discards inherent correlations between those patches, so as to learns redundant filters (the same or similar ones with translated versions). Owing to the convolutional property, the dictionary learned from CSC achieves signal global coherence, tending to be more representative. It should be emphasized that the convolutional-based deep learning, evolving from pioneering work~\cite{lecun1998gradient,kavukcuoglu2010learning,krizhevsky2012imagenet}, has shown its extraordinary success in a broad range of high-level image understanding applications. This supervised learning based approach refers to one type of discriminative models, whereas, CSC is a kind of generative model, which, in general, is less task-specific than discriminative models.

To solve the optimization problems formulated in CSC, Zeiler et al.~\cite{zeiler2010deconvolutional} iteratively solves two subproblems (updating sparse codes and updating filters) using gradient decent in the form of convolutional operations, while performing convolution in spatial domain is computationally expensive. Recent algorithms tackles the problem by exploiting Parseval's theorem to express the spatial convolution by multiplication in the frequency domain and using Alternating Direction Method of Multipliers (ADMM)~\cite{boyd2011distributed} to separate the non-smooth terms in the optimization problem, and they show tremendous improvements over prior spatial-domain solvers with respect to running time~\cite{bristow2013fast,heide2015fast,wohlberg2016efficient}. Most of the prior work learns the dictionary filters in a batch mode, which indicates that all training signals are involved in every training iteration, and this restricts it from applying to large datasets or streaming data. 

In contrast to batch mode learning, online learning~\cite{shalev2012online} is a well established strategy which processes a single or a portion (mini-batch) of the whole data at each training step, and incrementally updates model variables. Herein, the required memory and computing sources are only dependent on the sample size in every observation, independent of the training data size. It alleviates the scalability issue arisen in batch approaches, and the convergence of the algorithm was firstly analyzed using stochastic approximation tools~\cite{bottou1998online}. Bottou et al.~\cite{bousquet2008tradeoffs} further showed better generalization performance of the stochastic algorithms than standard gradient descent on large scale learning systems. Later on, online learning strategies were synergetic with sparse coding, which was then scaled up to learn dictionary from millions of training samples~\cite{mairal2009online,mairal2010online}. More recently, Liu et al.~\cite{liu-2018-first} proposed an online learning based CSC model, and analyzed different solvers and strategies when implementing it.

{\bfseries Contributions.} We mainly make three contributions in this work. First of all, we introduce a randomization strategy in the subproblem of updating sparse codes and solve the entire problems in spatial domain. It demonstrates that the proposed stochastic spatial-domain solver, with a reasonably selected subsampling probability, outperforms the state-of-the-art frequency-domain solvers with less computation time (per iteration), and also achieves faster convergence (less number of iterations). We then formulate an online-learning version of the proposed algorithm, and it shows dramatic runtime improvement over precedent online CSC methods, as well as producing comparable outcomes. At last, we demonstrate its capability to learn the over-complete dictionary from thousands of images, and analyze the effectiveness of the learned over-complete dictionary, which has not been reported or analyzed by precedent CSC work.