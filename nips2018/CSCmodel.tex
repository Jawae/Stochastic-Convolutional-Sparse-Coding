\section{Convolutional Sparse Coding (CSC)}
Specifically, the CSC problem has the form
\begin{equation}
\begin{split}
    \minimize{\filter,\code} & \frac{1}{2}\|\signal - \sum_{k=1}^{K} \filter_k * \code_k \|_2^2 + \lambda \sum_{k=1}^{K}\| \code_k \|_1 \\
    \text{subject to} & ~ \|\filter_k\|^2_2 \leq 1 ~~ \forall k \in \{1,\dots,K\},
\end{split}
\end{equation}
where $\signal \in \mathbb{R}^D$ is a $D$-dimensional signal or a vectorized image\footnote{In this manuscript, we work on 2D images.}, $\filter_k \in \mathbb{R}^M$ is the $k$-th dictionary, $\code_k\in \mathbb{R}^D$ is the sparse code associated with that dictionary,  $\lambda>0$ is a sparsity inducing penalty parameter, $K$ is the number of dictionary filters, and $*$ is the convolution operator. The above model will be applied to all the training images.

The modern approaches exploit Parseval's theorem and introduce two slack variables to separate the non-smooth $L_1$ penalty term and the $L_2$ constraints, making it feasible to be efficiently computed in frequency domain, and to apply splitting strategy to formulate the problem into ADMM framework, where the two subproblems (updating $\code$ and updating $\filter$) are jointly solved by coordinate descent~\cite{bristow2013fast,heide2015fast,wohlberg2016efficient}. The model and existing frequency solvers suffer from several issues:

\begin{itemize}
  \item While CSC overcomes the independence assumption held in patch-based learning algorithms, far more variables ($K$ times more) are introduced to represent a single image to compensate for this. This creates more severe memory and computational burdens.
  
  \item We observe through experiments that the vast majority of the entries of the reconstructed sparse codes do not provide useful information about the represented image. For $K=100$, 99.5\% entries are not informative. This indicates that the subproblem for updating $\code$ solves a highly sparse LASSO problem. Transforming the problem into frequency domain imposes restriction on the use of optimization strategies for this very specific problem.
 
  \item While prior work shows its efficiency in solving the CSC problem in the frequency domain, this is only applicable for updating $\code$, and does not hold for updating $\filter$. The dictionary filters usually have much smaller spatial support than the dimension size of the sparse codes ($M \ll D$). However, in order to tackle the problem in the frequency domain, it requires to process the $\filter$-subproblem in the same dimensions as the sparse codes, and then project the results onto its small spatial support.
\end{itemize}