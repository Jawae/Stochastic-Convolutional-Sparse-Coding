\section{Conclusions}
In this work, we present a novel stochastic subsampling strategy for updating sparse codes and it significantly improves the runtime performance to solve the CSC problem in spatial domain, which applies to both batch mode and online-learning mode, as well as generating effective and proven outcomes. Since the subproblem of updating the code is a highly sparse LASSO, other specific optimization strategies can be applied to further accelerate the computation, for instance the idea proposed in~\cite{johnson2017stingycd}, which solves the LASSO problem by coordinate descent and skips unnecessary updates by the method of safe screening~\cite{ghaoui2012Swfe}. Moreover, instead of employing uniform subsampling strategy, alternative subsampling strategies can be studied in the future for a gain of learning efficiency, for instance an importance or correlation based sampling strategy which considers the dependency relationship between the images and the sparse codes. Notice that transforming the problem into Fourier domain cannot benefit from these kinds of speedup strategies. In addition, we have shown the capability of the developed online algorithm to learn representative over-complete dictionary from large datasets, and the availability of the dictionary is further verified by the application of image inpainting. It can be foreseen that this capability has widespread applications in the audio and image related tasks, and higher dimensional signal processing.