\section{Conclusions}
In this work, we present a novel stochastic subsampling strategy for
solving the CSC problem in the spatial domain. This method exploits 
the sparsity of the \changed{over-parameterized} model and improves the runtime performance over the prior
frequency-domain solvers, which applies to both batch mode and
online-learning mode.  The proposed algorithm, for the first time,
demonstrates the feasibility that tackling the CSC problem in spatial
domain while still holding, or even improving the runtime
efficiency. Since the subproblem of updating the code is a highly
sparse LASSO, other specific optimization strategies can be applied to
further accelerate the computation, for instance the idea proposed
in~\cite{johnson2017stingycd}, which solves the LASSO problem by
coordinate descent and skips unnecessary updates using the method of
safe screening~\cite{ghaoui2012Swfe}. It is worth emphasizing that
Frequency-domain methods cannot benefit from these kinds of speedup
strategies.

%In the future, alternative subsampling strategies can be studied to
%improve learning efficiency. Instead of employing uniform subsampling
%strategy, for instance one could employ an importance or correlation
%based sampling strategy that considers the inter-dependency between the
%signals and the sparse codes.

We have also shown the capability of the developed online algorithm to
learn representative and meaningful over-complete dictionary from
arbitrary large datasets, and the availability of the dictionary is
further verified by the application of image inpainting. It can be
foreseen that this capability has widespread applications in audio
and image related tasks, and higher dimensional signal processing.


% --- DO NOT DELETE ---
% Local Variables:
% mode: latex
% mode: flyspell
% mode: TeX-PDF
% End:

